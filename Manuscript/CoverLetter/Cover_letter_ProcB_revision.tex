\documentclass[11pt,letterpaper]{letter}

\usepackage{RPKLetter}
\usepackage{url}
\usepackage{changepage}
\usepackage{multicol}

\begin{document}


%	TO ADDRESS

\begin{letter}
{}
%\\
%	LETTER CONTENT

\opening{}

\noindent Dear Editors,
\vspace{0.05in}

Please accept the enclosed revised manuscript, \textit{Environmental DNA Metabarcoding Reveals Winners and Losers of Global Change in Coastal Waters}, for consideration in Proceedings of the Royal Society B. The original submission number was RSPB-2019-2901. 

We have taken the opportunity to revise the manuscript significantly in response to reviewers' comments. In addition, of course, the COVID crisis delayed the revision somewhat.

In the piece, we report genetic data taken from a ca. 18-month time-series of water samples (N = 227), from along a spatial gradient in sea-surface temperature, pH, and other environmental parameters. After isolating the effects of these water conditions on eukaryotic communities, we derive models that describe species-specific responses to changes in environment. We then use these models to forecast community change in a future ocean scenario, finding large shifts in community composition that will likely result in fundamental changes to these ecosystems. We are very excited about this work.

A central problem in understanding the effects of global change -- and indeed, ecology generally -- is the usual tradeoff between depth and breadth: traditional approaches offer either an in-depth assessment of one or few species (e.g., laboratory experiments) or a shallow assessment of many species (e.g., meta-analyses and modeling).  Here, we use eDNA metabarcoding to overcome this challenge; our empirical dataset consists of hundreds of species sampled from along an environmental gradient that captures the conditions many nearshore habitats will experience in the coming decades. 

Most strikingly, we see a decline in diatom richness and abundance and a rise in dinoflagellates across our study area, portending major shifts in community composition and the marine food web during this century. 

We think this work represents a significant advance, using eDNA sequencing data to move beyond proofs-of-concept and into the realm of empirical ecology. By linking this emerging method to water chemistry and oceanographic projections, we offer a way of characterizing community-level changes that are otherwise impossible to capture. Moreover, the technique is broadly useful in the study of ecological change in the ocean; importantly, it complements data from model sites such as CO$_2$ vents by revealing patterns in non-model systems that are the rule in the rest of the world. 

We think the manuscript has a wide appeal for ecologists, geneticists, marine food-web scholars, and scientists interested in the different ways we can forecast changes in marine ecosystems with climate change. For such a potentially wide audience, Proceedings B is a perfect fit and we want to see our work published here. We sincerely hope you share our view and we are looking forward to hearing from you,





 
\closing{Cheers,}

%----------------------------------------------------------------------------------------
\end{letter}
\end{document}
